\section{Konklusion}
Formålet med dette projekt var at lave et aktivt festival anlæg. Gennem processen mener gruppen at målet med et funktionelt aktivt anlæg er nået.  \\ \\
Frekvensresponset for selve anlægget giver et forholdsvis flat respons, som stemmer overens med målet for optimeringen ift standartfilteret. Herigennem gav kabinettet nogle udtilsigtede udjævnheder jævnfør frekvensresponset, dog med et positivt basrefleks, som blev valgt at beholde. Dette skyldes i høj grad også hvad anlægget skal bruges til, og festival gængere ofte søger et højt lydtryk i de lave frekvenser. \\ \\
Gennem projektet valgte gruppen at lave to konfigurationer som skulle optimere to forskellige musikgenrer. Gennem lyttetest, er gruppen blevet opmærksom på hvilke konfigurationer, som havde den ønskede effekt, og hvilke som ikke havde. \\ \\
Gruppen har gennem processen fået erfaring omkring målinger af højtalere, akkustiske overvejelser og udfordringer. Derigennem er der også opnået større indsigt i optimering af hhv. standart filter, så vel som optimering filtrene jazz og rock. \\ 
Hertil har gruppen også stiftet kendskab til lyttetest, hvor præcision og målrettethed, er den vigtiste faktor, og herved også hvilken tidskrævende proces en lyttetest kan være.  