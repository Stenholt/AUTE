\section{Konklusion}
Formålet med dette projekt var at lave et aktivt festival anlæg og optimere det til at lyde bedst muligt. Gennem processen mener gruppen at målet med et optimeret funktionelt aktivt anlæg er nået. 

Frekvensresponset for selve anlægget giver et forholdsvis flad respons, som stemmer overens med målet for den optimerede konfiguration   ift.  standart konfiguration. Kabinettet gav nogle ændringer på frekvensresponset,  men generelt blev resultatet fint og med et positivt bidrag fra kabinettets basrefleks, som blev valgt at beholde åben. Dette skyldes i høj grad på grund af anvendelsen af anlægget, hvor især festival gængere ofte søger et højt lydtryk i de lave frekvenser.

Gennem projektet valgte gruppen at lave to konfigurationer, som skulle optimere to forskellige musikgenrer. Gennem lyttetest, er gruppen blevet opmærksom på hvilke konfigurationer, som havde den ønskede effekt, og hvilke som ikke havde helt den ønskede effekt. Rock konfigurationen var en succes mens Jazz konfigurationen fejlede.    

Gruppen har gennem processen fået erfaring omkring målinger af højtalere, akustiske overvejelser og udfordringer. Derigennem er der også opnået større indsigt i optimering af hhv.det optimerede standart filter, så vel som de to optimeringsfiltre for jazz og rock.
Hertil har gruppen også stiftet kendskab til lyttetest, hvor præcision og målrettethed, er en af de vigtigste faktorer, og herved også hvilken tidskrævende proces en lyttetest kan være.  

Alt i alt har projektet været en succes og samtidig særdeles lærerigt på mange måder.