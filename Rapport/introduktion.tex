\section{Introduktion}

Mange festivalgængerne slæber hvert år om sommeren et stort festivalanlæg med til ”campen” på diverse festivaler. Ofte er disse anlæg hjemmebyggede med analoge delefiltre og uden nogen form for nogen optimering udover måske niveau justering, da dette hurtigt bliver besværligt. Ved at i stedet at lave en aktiv højtaler kan man derimod rimeligt simpelt justere og optimere lyden, som man ønsker, blot ved at ændre på nogle digitale filtre på en DSP. Man behøver heller ikke bruge meget tid på at lodde delefiltre, da dette også kan klares hurtigt og ændres igen på en DSP. 

Ideen er at man i princippet kan tage nogle billige enheder, putte dem i et kabinet og derefter korrigere og optimere højtaleren på en DSP til at lyde meget bedre end udgangspunktet. En billig enhed kan derfor komme til at opfører sig som en dyrere og bedre enhed, og tilsvarende kan en dyr enhed opføre som en meget dyr enhed. Dette er en god egenskab for fx et festivalanlæg, hvor man gerne vil have masser af ”god” lyd til billige penge.  

Dette projekt omhandler derfor designet af en stereo 2-vejs højtaler med aktiv crossover og optimering på en DSP. Højtaleren er tiltænkt som et hjemmebygget festivalanlæg, hvor der benyttes nogle gamle enheder fra en defekt højtaler og kabinetterne fra et andet gammelt højtaler par købt i en genbrugsforretning til 50 kr. Formålet er at sammensætte dette system og optimere det til at lyde bedste muligt ved hjælp af en DSP. Derudover ønskes der mulighed for yderlig at kunne vælge 2 konfigurationer til henholdsvis Rock og Jazz som equaliseres med FIR filtre på DSP’en. Disse konfigurationer evalueres subjektivt med en kontrolleret lyttetest på en håndfuld personer.
